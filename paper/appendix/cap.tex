\section{Caselaw Access Project (CAP)}
\label{appendix:cap}

This appendix provides details about the Caselaw Access Project (CAP) dataset, including its collection methodology, data statistics, and examples of the data. The Caselaw Access Project is a collaborative effort led by Harvard Law School's Library Innovation Lab to digitize and freely share all U.S. case law.

\subsection{Dataset Overview}

The CAP dataset contains court opinions from U.S. state and federal courts, providing a comprehensive repository of legal precedents. The corpus includes published opinions from all federal courts and state appellate courts, covering nearly 6.92 million cases from the 1700s to the present day. This represents one of the most comprehensive collections of historical and contemporary U.S. case law available for research and analysis.

\subsection{Data Processing Statistics}

Based on the counts files in the KL3M project, the CAP dataset has been processed through multiple stages:

\begin{table}[h]
\centering
\begin{tabular}{|l|r|}
\hline
\textbf{Processing Stage} & \textbf{Number of Documents} \\
\hline
Documents (Initial Collection) & 6,919,296 \\
Representations (Processed) & 6,919,272 \\
Parquet (Final Format) & 6,919,272 \\
\hline
\end{tabular}
\caption{CAP Dataset Document Counts by Processing Stage}
\label{tab:cap_counts}
\end{table}

As shown in Table \ref{tab:cap_counts}, nearly all collected documents (over 99.99\%) were successfully processed through each stage of the pipeline. The minimal difference between the initial collection count and the final processed count (24 documents) indicates the robustness of the processing pipeline for this dataset.

\subsection{Collection Methodology}

The CAP dataset was collected from the Harvard Law School's Case Law Access Project API and static archive. The collection process involved the following steps:

\begin{enumerate}
  \item A list of ZIP file URLs was compiled from the static.case.law archive, stored in the \texttt{zip\_urls.txt.gz} file within the KL3M codebase. Each URL points to a ZIP file containing multiple court opinions from a specific reporter.
  
  \item For each ZIP file URL, the collection pipeline:
  \begin{itemize}
    \item Downloads the ZIP archive from static.case.law
    \item Extracts both the HTML files (containing the case text) and the corresponding JSON metadata files
    \item Embeds each HTML fragment into a proper HTML document structure
    \item Creates a Document object with the following metadata:
    \begin{itemize}
      \item Unique ID from the CAP dataset
      \item Case name as the title
      \item Format identifier (text/html)
      \item Description (case name)
      \item Source URL (https://static.case.law/)
      \item License information (CC0 1.0 Universal)
      \item Content hash for integrity verification
      \item Case-specific metadata (court, jurisdiction, date, etc.)
    \end{itemize}
    \item Uploads each document to the KL3M storage system
  \end{itemize}
\end{enumerate}

The collection methodology leverages the publicly available data from the Case Law Access Project, which digitized over 40 million pages of U.S. court decisions in collaboration with Ravel Law. The dataset is licensed under CC0 1.0 Universal, placing it in the public domain. This ensures that the entire corpus is freely available for research, analysis, and reuse without copyright restrictions.

\subsection{Content Examples}
% TODO: Add representative examples of court opinions
