\section{Regulations.gov Documents}
\label{appendix:reg_docs}

This appendix provides details about the Regulations.gov Documents dataset, which consists of documents submitted to the federal rulemaking portal, including public comments, proposed rules, supporting materials, and regulatory attachments.

\subsection{Processing Statistics}

Table~\ref{table:reg-docs-stats} shows the processing statistics for the Regulations.gov Documents dataset.

\begin{table}[h]
\centering
\begin{tabular}{|l|r|r|r|}
\hline
\textbf{Processing Stage} & \textbf{Document Count} & \textbf{\% of Documents} & \textbf{\% of Initial} \\
\hline
Documents (Initial Collection) & 1,279,349 & 100\% & 100\% \\
Representations (Processed) & 811,163 & 63.4\% & 63.4\% \\
Parquet (Final Format) & 785,062 & 61.4\% & 61.4\% \\
\hline
\end{tabular}
\caption{Processing statistics for the Regulations.gov Documents dataset.}
\label{table:reg-docs-stats}
\end{table}

The Regulations.gov Documents dataset shows a processing success rate of 61.4\% from initial collection to final format. This relatively lower conversion rate compared to some other datasets in the KL3M collection is likely due to the diverse nature of submitted documents, which include various file formats, scanned images, and occasionally corrupted files submitted through the public comment system.

\subsection{Collection Methodology}

The Regulations.gov Documents dataset is collected using the official Regulations.gov API v4. The collection methodology involves:

\begin{enumerate}
    \item Accessing the Regulations.gov API using an API key (required for all operations)
    \item Systematically retrieving documents by date, starting from January 1, 2010
    \item For each date, retrieving all documents posted on that date using pagination to handle large result sets
    \item For each document, retrieving detailed information including all associated file formats and attachments
    \item Downloading the content of each document and attachment from the provided file URLs
    \item Processing each document through the following steps:
    \begin{itemize}
        \item Checking if the document already exists in the collection to avoid duplication
        \item Determining the document's content type from response headers
        \item Extracting metadata such as title, posted date, and agency ID
        \item Generating a blake2b hash of the document content
        \item Creating a Document object with comprehensive metadata
        \item Storing the document in the KL3M collection
    \end{itemize}
\end{enumerate}

The collection process includes sophisticated rate limiting techniques to comply with API restrictions. The system dynamically adjusts request delays based on the rate limit information provided in API response headers, increasing or decreasing delays to optimize throughput while avoiding API rate limit errors.

\subsection{API Details and Rate Limiting}

The Regulations.gov API v4 enforces strict rate limits on requests. To manage these limits, the collection process:

\begin{enumerate}
    \item Tracks the number of remaining API requests through response headers
    \item Implements adaptive delay between requests:
    \begin{itemize}
        \item Decreases delay by 2.5\% when rate limit headroom increases
        \item Increases delay by 5\% when rate limit headroom decreases
        \item Maintains a slight decrease (0.5\%) when usage remains stable
        \item Enforces minimum and maximum delay bounds (2.0-6.5 seconds)
    \end{itemize}
    \item Handles API errors and temporary failures with appropriate retries
\end{enumerate}

This adaptive approach ensures efficient collection while maintaining compliance with the API's terms of service.

\subsection{Document Types and Content}

The Regulations.gov Documents dataset encompasses a wide variety of document types related to the federal rulemaking process:

\begin{enumerate}
    \item \textbf{Public Comments}
    \begin{itemize}
        \item Feedback from individuals, organizations, and stakeholders on proposed regulations
        \item Personal statements, expert opinions, and grassroots responses
        \item Form letters and mass comment campaigns
    \end{itemize}
    
    \item \textbf{Supporting Materials}
    \begin{itemize}
        \item Scientific studies and research papers
        \item Technical analyses and impact assessments
        \item Economic evaluations and cost-benefit analyses
        \item Market surveys and industry data
    \end{itemize}
    
    \item \textbf{Attachments and Exhibits}
    \begin{itemize}
        \item Data tables, charts, and graphs
        \item Photographs and diagrams
        \item Engineering specifications and technical drawings
        \item Affidavits and witness statements
    \end{itemize}
    
    \item \textbf{Regulatory Documents}
    \begin{itemize}
        \item Proposed rules and regulatory text
        \item Supplementary information and background materials
        \item Response to comments documents
        \item Regulatory flexibility analyses
    \end{itemize}
\end{enumerate}

Documents are primarily stored in PDF format, though the original submission may have been in various formats. The Regulations.gov system typically converts uploads to PDF for consistent public access.

\subsection{Metadata Structure}

Each document in the dataset contains rich metadata including:

\begin{enumerate}
    \item \textbf{Basic Identifiers}
    \begin{itemize}
        \item Document ID (e.g., EPA-HQ-OPP-2007-1024-0726)
        \item Unique identifier URL (e.g., https://downloads.regulations.gov/EPA-HQ-OPP-2007-1024-0726/content.pdf)
    \end{itemize}
    
    \item \textbf{Temporal Information}
    \begin{itemize}
        \item Posted date (when the document was made available on Regulations.gov)
    \end{itemize}
    
    \item \textbf{Authorship and Publishing Details}
    \begin{itemize}
        \item Agency ID (the federal agency responsible for the regulatory action)
        \item Document title
        \item Content type (usually application/pdf)
    \end{itemize}
    
    \item \textbf{Technical Metadata}
    \begin{itemize}
        \item Content size in bytes
        \item Cryptographic hash (blake2b) for integrity verification
        \item Original API response containing full document details
    \end{itemize}
\end{enumerate}

The dataset preserves all original metadata from the Regulations.gov API, enabling researchers to access the complete context of each document.

\subsection{Legal Status}

The Regulations.gov Documents dataset is made available according to the Regulations.gov User Notice, which permits unrestricted third-party use of the materials. Key aspects of the legal status include:

\begin{enumerate}
    \item Public domain status for documents created by federal agencies (under 17 U.S.C. § 105)
    \item Public records status for submitted comments and materials
    \item Attribution to the eRulemaking Program Management Office as the publisher
    \item Attribution to the specific federal agency as the creator for agency-produced documents
\end{enumerate}

While the documents themselves are freely available for use, researchers should be aware that some submitted materials might contain third-party content with different copyright status.

\subsection{Research Value}

The Regulations.gov Documents dataset offers unique research opportunities:

\begin{enumerate}
    \item \textbf{Regulatory Process Analysis}
    \begin{itemize}
        \item Understanding public participation in rulemaking
        \item Studying the evolution of regulatory text based on public input
        \item Analyzing the quality and impact of public comments
    \end{itemize}
    
    \item \textbf{Natural Language Processing Applications}
    \begin{itemize}
        \item Sentiment analysis of public comments
        \item Topic modeling of regulatory discussions
        \item Extraction of technical arguments and evidence
        \item Classification of comment types and stakeholder groups
    \end{itemize}
    
    \item \textbf{Policy Impact Studies}
    \begin{itemize}
        \item Measuring public response to proposed regulations
        \item Identifying key concerns across different regulatory domains
        \item Tracking changes in regulatory approach over time
    \end{itemize}
\end{enumerate}

The dataset provides a comprehensive view of public participation in the U.S. federal regulatory process, capturing diverse viewpoints from individuals, businesses, non-profits, and other stakeholders.

\subsection{Relationship to Other Datasets}

The Regulations.gov Documents dataset complements other regulatory datasets in the KL3M collection:

\begin{enumerate}
    \item \textbf{Federal Register} (Section~\ref{appendix:fr}) - Contains the official published proposed and final rules, while Regulations.gov Documents contains the public comments and supporting materials
    
    \item \textbf{Electronic Code of Federal Regulations} (Section~\ref{appendix:ecfr}) - Represents the final codified regulations, while Regulations.gov Documents shows the development process and public input
    
    \item \textbf{GovInfo} (Section~\ref{appendix:govinfo}) - Provides broader government publications, while Regulations.gov Documents focuses specifically on regulatory development materials
\end{enumerate}

Together, these datasets provide a comprehensive view of the U.S. regulatory landscape, from initial proposal through public comment to final implementation.