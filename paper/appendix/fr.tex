\section{Federal Register}
\label{appendix:fr}

This appendix provides details about the Federal Register dataset, including its collection methodology, data statistics, and examples of the data. The Federal Register is the official journal of the United States government, containing federal agency regulations, proposed rules, public notices, executive orders, proclamations, and other presidential documents.

\subsection{Dataset Overview}

The Federal Register dataset contains the complete collection of documents published in the Federal Register since 1995. It serves as the official source for federal regulations, notices, and presidential documents, making it essential for legal research, regulatory compliance, and policy analysis. The dataset provides access to the full text of proposed rules, final rules, notices, and presidential documents published daily by Federal agencies.

This dataset is particularly valuable because it represents the official publication through which regulatory changes are announced, public comments are solicited, and new federal requirements are established. It covers all executive branch agencies and provides a comprehensive record of the federal regulatory process.

\subsection{Data Processing Statistics}

Based on the counts files in the KL3M project, the Federal Register dataset has been processed through multiple stages:

\begin{table}[h]
\centering
\begin{tabular}{|l|r|}
\hline
\textbf{Processing Stage} & \textbf{Number of Documents} \\
\hline
Documents (Initial Collection) & 3,396,818 \\
Representations (Processed) & 3,396,455 \\
Parquet (Final Format) & 3,396,389 \\
\hline
\end{tabular}
\caption{Federal Register Dataset Document Counts by Processing Stage}
\label{tab:fr_counts}
\end{table}

As shown in Table \ref{tab:fr_counts}, the Federal Register dataset maintained exceptional consistency across all processing stages. From the initial collection to the representation stage, only 363 documents were lost (99.99\% retention). The final parquet conversion stage had a minimal additional loss of 66 documents. Overall, 99.99\% of the originally collected documents were successfully processed through the entire pipeline, demonstrating the high quality of both the source data and the processing methodology.

\subsection{Collection Methodology}

The Federal Register dataset was collected using the official Federal Register API provided by the Government Publishing Office (GPO) and the National Archives and Records Administration (NARA). The collection process involved several sophisticated steps:

\begin{enumerate}
  \item \textbf{Date-Based Retrieval}: The system iterates through all dates from January 1, 1995 (\texttt{FR\_MIN\_DATE}) to the present day, retrieving all documents published on each date.
  
  \item \textbf{Document Discovery}: For each date, the system queries the Federal Register API (\texttt{/api/v1/documents.json}) with a comprehensive set of metadata fields to identify all documents published on that date.
  
  \item \textbf{Multiple Format Collection}: For each document identified, the system retrieves all available formats:
  \begin{itemize}
    \item \textbf{XML}: Full XML content from the \texttt{full\_text\_xml\_url} endpoint, providing structured data about the document
    \item \textbf{Plain Text}: Raw text content from the \texttt{raw\_text\_url} endpoint
    \item \textbf{PDF}: PDF version from the \texttt{pdf\_url} endpoint
    \item \textbf{HTML}: HTML content from the \texttt{body\_html\_url} endpoint, which includes the formatted document
  \end{itemize}
  
  \item \textbf{Document Creation}: For each format of each document, the system creates a Document object with comprehensive metadata:
  \begin{itemize}
    \item Unique ID in the format \texttt{document\_number/extension}
    \item Document content in the specific format
    \item Size and cryptographic hash (blake2b) for integrity verification
    \item Title and abstract from the original document
    \item Publication date
    \item Agency information as creator metadata
    \item Subject information from document type, subtype, and topics
    \item Citation and bibliographic information
    \item Complete metadata from the API response in the extra field
  \end{itemize}
  
  \item \textbf{Storage}: Each format of each document is uploaded to the KL3M storage system.
\end{enumerate}

The collection process includes sophisticated error handling to ensure that even if retrieval of one format fails, other formats can still be processed and stored. It also applies rate limiting to respect the API's usage policies and avoid overwhelming the government servers.

\subsection{Document Types}

The Federal Register contains several distinct types of documents, each serving a different purpose in the federal regulatory process:

\begin{itemize}
  \item \textbf{Proposed Rules}: Documents announcing draft regulations and soliciting public comments
  \item \textbf{Rules}: Final regulations with the force of law
  \item \textbf{Notices}: Announcements of meetings, information collections, grant applications, etc.
  \item \textbf{Presidential Documents}: Executive orders, proclamations, and other presidential actions
  \item \textbf{Sunshine Act Meetings}: Announcements of federal agency meetings open to the public
  \item \textbf{Corrections}: Documents correcting errors in previously published materials
\end{itemize}

\subsection{Legal Status}

As noted in the source code, the Federal Register dataset is "Not subject to copyright under 17 U.S.C. 105," which means that the content is in the public domain as it was created by the federal government. This makes the dataset freely available for research, analysis, and redistribution without copyright restrictions.

\subsection{Data Transformation}

The collection process includes the capability to transform XML documents using an XSLT stylesheet (\texttt{fedregister.xsl}). This transformation converts the structured XML content into HTML format, preserving the document's structure while making it more accessible for viewing and analysis. The XSLT transformation carefully preserves elements such as:

\begin{itemize}
  \item Document headers and metadata
  \item Agency and action information
  \item CFR references and regulatory text
  \item Structural elements like sections, paragraphs, and tables
  \item Formatting for special elements like footnotes, appendices, and signatures
\end{itemize}

This transformation ensures that the rich structure of Federal Register documents is maintained while being made accessible in standard web formats.
