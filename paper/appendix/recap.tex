\section{RECAP Archive}
\label{appendix:recap}

This appendix provides details about the RECAP Archive dataset, which consists of documents and filings from the federal judiciary's Public Access to Court Electronic Records (PACER) system that have been collected by the Free Law Project's RECAP initiative.

\subsection{Processing Statistics}

Table \ref{table:recap-stats} shows the processing statistics for the RECAP Archive dataset.

\begin{table}[h]
\centering
\begin{tabular}{|l|r|r|r|}
\hline
\textbf{Processing Stage} & \textbf{Document Count} & \textbf{\% of Documents} & \textbf{\% of Initial} \\
\hline
Documents (Initial Collection) & 16,762,471 & 100\% & 100\% \\
Representations (Processed) & 14,423,347 & 86.0\% & 86.0\% \\
Parquet (Final Format) & 14,265,800 & 85.1\% & 85.1\% \\
\hline
\end{tabular}
\caption{Processing statistics for the RECAP Archive dataset.}
\label{table:recap-stats}
\end{table}

\subsection{Collection Methodology}

The RECAP Archive dataset is collected from the Free Law Project's RECAP initiative, which allows users of the PACER system to contribute documents they have accessed to a public archive. The collection methodology involves:

\begin{enumerate}
    \item Accessing the public Court Listener S3 bucket (com-courtlistener-storage) where the RECAP project stores the contributed documents.
    \item Iterating through all objects with the prefix "recap/" in the S3 bucket.
    \item Processing each document, extracting metadata, and deduplicating based on content hashes (using blake2b).
    \item For PDF files, extracting additional metadata using the pypdfium2 library, including page count and embedded document metadata.
    \item For XML files (specifically docket XML files), extracting basic metadata.
    \item Storing the documents with appropriate metadata including court information, document type, and source attribution.
\end{enumerate}

The collection process includes court identification, which maps court identifiers (e.g., "ca1", "nysd") to their full names (e.g., "United States Court of Appeals for the First Circuit", "United States District Court for the Southern District of New York").

\subsection{Document Types}

The RECAP Archive dataset includes two primary document types:

\begin{enumerate}
    \item \textbf{PDF documents} - These include various court filings such as complaints, motions, orders, opinions, and other legal documents that are filed in federal courts. The PDFs are processed to extract metadata such as page count, title, author, and other embedded information.
    
    \item \textbf{XML docket files} - These files, ending with ".docket.xml", contain structured information about court cases, including case metadata, party information, and a chronological list of docket entries describing the documents filed in the case.
\end{enumerate}

Both document types provide valuable information about federal court cases, with the PDF files containing the actual text of legal documents and the XML files providing structured case data.

\subsection{Legal Status}

The RECAP Archive dataset is made available under CC0/Public Domain. While federal court documents are generally considered public domain under 17 U.S.C. § 105, which excludes works of the U.S. Government from copyright protection, the PACER system itself charges access fees. The RECAP project makes these public domain documents freely available by collecting them from users who have already paid for access.

According to the Free Law Project, "court opinions are not copyrightable and thus remain in the public domain. Court orders and other documents created by courts as part of their official duties are also not copyrightable." Documents submitted by parties in court cases may potentially be subject to copyright, though they are considered publicly accessible once filed in court.

\subsection{Distinguishing RECAP Archive from RECAP Documents}

It's important to note the distinction between the "recap" dataset described in this section and the related "recap\_docs" dataset (covered in the next section). The main differences are:

\begin{enumerate}
    \item The RECAP Archive ("recap") contains primarily docket entries and main case documents retrieved from PACER.
    \item The RECAP Documents ("recap\_docs") contains attachment files associated with court documents, including various file formats like .doc, .docx, .mp3, .pdf, and .wpd.
\end{enumerate}

Together, these datasets provide a comprehensive view of federal court cases and filings.