\section{GovInfo}
\label{appendix:govinfo}

This appendix provides details about the GovInfo dataset, which contains official publications from all three branches of the Federal Government. GovInfo is a service of the United States Government Publishing Office (GPO), providing free public access to a comprehensive repository of government information.

\subsection{Processing Statistics}

Table~\ref{table:govinfo-stats} shows the processing statistics for the GovInfo dataset.

\begin{table}[h]
\centering
\begin{tabular}{|l|r|r|r|}
\hline
\textbf{Processing Stage} & \textbf{Document Count} & \textbf{\% of Documents} & \textbf{\% of Initial} \\
\hline
Documents (Initial Collection) & 14,655,232 & 100\% & 100\% \\
Representations (Processed) & 13,353,022 & 91.1\% & 91.1\% \\
Parquet (Final Format) & 11,144,653 & 76.0\% & 76.0\% \\
\hline
\end{tabular}
\caption{Processing statistics for the GovInfo dataset.}
\label{table:govinfo-stats}
\end{table}

The GovInfo dataset shows a good initial processing success rate of 91.1\% from raw documents to representations, with some additional reduction to 76.0\% in the final parquet stage. This processing pattern reflects the diverse nature of government publications, which include complex document formats, nested structures, and varying content quality. The dataset still maintains over 11 million documents in the final format, making it one of the largest components of the KL3M collection.

\subsection{Collections}

The KL3M dataset includes content from multiple collections available in GovInfo. Some documents may appear in multiple collections, as indicated by the semicolon-delimited collection IDs in the source data. The primary collections are:

\begin{itemize}
  \item \textbf{BILLS} -- Congressional Bills
  \item \textbf{BUDGET} -- Budget of the United States Government
  \item \textbf{CCAL} -- Congressional Calendars
  \item \textbf{CDIR} -- Congressional Directory
  \item \textbf{CDOC} -- Congressional Documents
  \item \textbf{CHRG} -- Congressional Hearings
  \item \textbf{CMR} -- Commerce Business Daily
  \item \textbf{COMPS} -- Compilations of Presidential Documents
  \item \textbf{CPD} -- Congressional Pictorial Directory
  \item \textbf{CPRT} -- Congressional Committee Prints
  \item \textbf{CREC} -- Congressional Record
  \item \textbf{CRECB} -- Congressional Record Bound
  \item \textbf{CRI} -- Congressional Record Index
  \item \textbf{CRPT} -- Congressional Reports
  \item \textbf{CZIC} -- Coastal Zone Information Center
  \item \textbf{ECONI} -- Economic Indicators
  \item \textbf{ERIC} -- Education Resources Information Center
  \item \textbf{ERP} -- Economic Report of the President
  \item \textbf{GAOREPORTS} -- Government Accountability Office Reports
  \item \textbf{GOVMAN} -- Government Manual
  \item \textbf{GOVPUB} -- Government Publications
  \item \textbf{GPO} -- Government Publishing Office Collections
  \item \textbf{HJOURNAL} -- House Journal
  \item \textbf{HMAN} -- House Manual
  \item \textbf{HOB} -- History of Bills
  \item \textbf{LSA} -- List of CFR Sections Affected
  \item \textbf{PAI} -- Privacy Act Issuances
  \item \textbf{PLAW} -- Public and Private Laws
  \item \textbf{PPP} -- Public Papers of the Presidents
  \item \textbf{SERIALSET} -- Serial Set
  \item \textbf{SMAN} -- Senate Manual
  \item \textbf{SJOURNAL} -- Senate Journal
  \item \textbf{STATUTE} -- Statutes at Large
  \item \textbf{USCOURTS} -- United States Courts Opinions
\end{itemize}

\subsection{Cross-Collection Documents}

Some documents in GovInfo appear in multiple collections. The major cross-collection occurrences include:

\begin{itemize}
  \item Documents that appear in both their primary collection and the GPO collection (e.g., documents labeled as GPO;CDOC, GPO;CFR, GPO;CPRT, GPO;CRECB, GPO;CRPT, GPO;FR, GPO;SJOURNAL)
  \item Documents in the Serial Set that also belong to other collections (e.g., SERIALSET;CDOC, SERIALSET;CRPT, SERIALSET;HJOURNAL, SERIALSET;SJOURNAL)
  \item Congressional documents that appear in multiple collections (e.g., ERP;CDOC, HMAN;CDOC, SMAN;CDOC)
  \item Government publications in specific categories (e.g., GOVPUB;CHRG)
  \item Budget documents in multiple collections (SERIALSET;CDOC;BUDGET)
  \item Economic reports in multiple collections (SERIALSET;CRPT;ERP)
\end{itemize}

\subsection{Collection Methodology}

The GovInfo dataset is collected through the official GovInfo API provided by the U.S. Government Publishing Office. The collection methodology involves:

\begin{enumerate}
    \item \textbf{API Integration} -- Using the official GovInfo API with authenticated access through an API key
    
    \item \textbf{Systematic Collection} -- Retrieving documents through several complementary approaches:
    \begin{itemize}
        \item Collection-based retrieval targeting specific government collections
        \item Date-based retrieval covering documents from 1995 to the present
        \item Granule-level retrieval for collections with nested document structures
    \end{itemize}
    
    \item \textbf{Metadata Extraction} -- For each document:
    \begin{itemize}
        \item Retrieving comprehensive metadata including title, date issued, collection code, and government author
        \item Processing multi-collection documents to ensure proper categorization
        \item Handling temporal relationships between documents
    \end{itemize}
    
    \item \textbf{Content Download} -- Implementing a sophisticated download process that:
    \begin{itemize}
        \item Prioritizes text and PDF formats over ZIP archives when available
        \item Handles the on-demand generation of package content through API retry mechanisms
        \item Processes both package-level and granule-level content
    \end{itemize}
    
    \item \textbf{Duplicate Avoidance} -- Excluding collections already covered by separate specialized datasets (specifically CFR and FR)
\end{enumerate}

The collection process includes robust error handling for service interruptions, rate limiting compliance, and handling of dynamically generated content. The system implements intelligent caching to optimize API usage and ensure comprehensive coverage.

\subsection{Document Structure and Content Types}

The GovInfo dataset encompasses a diverse range of document types and formats reflecting the variety of government publications:

\begin{enumerate}
    \item \textbf{Document Organization}
    \begin{itemize}
        \item \textit{Packages} -- Top-level container for related documents
        \item \textit{Granules} -- Subdivisions within packages (e.g., sections of a bill, chapters of a report)
        \item \textit{Collections} -- Thematic groupings of documents by government function or source
    \end{itemize}
    
    \item \textbf{Content Formats}
    \begin{itemize}
        \item PDF documents for formatted reading
        \item Text formats for plain text content
        \item HTML/XML for structured online viewing
        \item ZIP archives containing multiple related documents
    \end{itemize}
    
    \item \textbf{Content Categories}
    \begin{itemize}
        \item Legislative materials (bills, reports, hearings)
        \item Executive publications (presidential papers, agency reports)
        \item Judicial documents (court opinions)
        \item Administrative documents (manuals, directories)
    \end{itemize}
\end{enumerate}

The system prioritizes the most accessible and useful formats, preferring text and PDF formats when available to ensure optimal processing in the KL3M pipeline.

\subsection{Metadata and Search Capabilities}

The GovInfo collection leverages rich metadata available through the API:

\begin{enumerate}
    \item \textbf{Identification Information}
    \begin{itemize}
        \item Package ID and Granule ID for unique document identification
        \item Collection codes indicating document source and category
        \item Document class distinguishing document types
    \end{itemize}
    
    \item \textbf{Temporal Information}
    \begin{itemize}
        \item Date issued for official publication date
        \item Last modified date for tracking updates
        \item Date ingested into the GovInfo system
    \end{itemize}
    
    \item \textbf{Legislative Context}
    \begin{itemize}
        \item Congress number for legislative session
        \item Session identifier for specific congressional period
        \item Branch indicator (Executive, Legislative, Judicial)
    \end{itemize}
    
    \item \textbf{Content Description}
    \begin{itemize}
        \item Title providing document summary
        \item Category classifying content type
        \item Government author attributing the creating entity
    \end{itemize}
\end{enumerate}

This rich metadata enables complex searches and filtering capabilities, allowing targeted retrieval based on various criteria including date ranges, government branches, document types, and content categories.

\subsection{Legal Status}

The GovInfo dataset is primarily in the public domain under 17 U.S.C. § 105, which states that works created by the federal government are not subject to copyright protection. As specified in the dataset metadata: "In general, GovInfo documents fall under 17 U.S.C. 101 and 17 U.S.C. 105 and are therefore not subject to copyright protection. Any incorporated material not covered by these provisions will be clearly marked with a notice per 17 U.S.C. 403."

The legal status ensures that:

\begin{enumerate}
    \item Government-produced content is freely available for research and analysis
    \item Third-party content incorporated into government documents is properly identified with copyright notices as required by law
    \item The dataset can be used without copyright restrictions for a wide range of applications
\end{enumerate}

\subsection{Research Applications}

The GovInfo dataset offers numerous research opportunities:

\begin{enumerate}
    \item \textbf{Legislative Analysis}
    \begin{itemize}
        \item Tracking the evolution of bills through Congress
        \item Analyzing voting patterns and legislative priorities
        \item Studying the language and structure of federal legislation
    \end{itemize}
    
    \item \textbf{Government Operations}
    \begin{itemize}
        \item Examining congressional hearing transcripts for policy development
        \item Analyzing budget documents for fiscal priorities
        \item Studying government reports for agency activities and recommendations
    \end{itemize}
    
    \item \textbf{Legal Research}
    \begin{itemize}
        \item Analyzing court opinions for legal precedents
        \item Studying the implementation of statutes and regulations
        \item Tracking changes in legal interpretation over time
    \end{itemize}
    
    \item \textbf{Political Science}
    \begin{itemize}
        \item Analyzing presidential communications and policy statements
        \item Studying congressional deliberations and debates
        \item Examining the relationship between branches of government
    \end{itemize}
\end{enumerate}

The comprehensive nature of this dataset, spanning all three branches of government, makes it particularly valuable for researchers interested in U.S. governance, policy development, and legal analysis.