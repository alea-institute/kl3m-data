\section{Copyright and Generative A.I.}
\subsection{Brief Overview of Copyright}
Copyright is sometimes described as a `bargain’ where creators receive, exclusive rights to their works for a specific period of time, in exchange for the public eventually gaining free access to those works after the copyright term expires.\cite{patterson1991nature} Although all materials will eventually reach the public domain, creators may, during their period of exclusive ownership, choose to make their works available to others via licensing at a scope of their choosing.  There is a wide continuum of popular copyright licenses including MIT, Apache, GNU GPL, AGPL as well as various flavors of Creative Commons licenses.\cite{metzger2015free}  Such licenses provide a wide degree of latitude for creators to control the scope of how their respective works are used.  

In the internet era, many individuals and organizations have chosen to make their otherwise copyrighted works available online for the scope delimited use of others. Through various platforms such as \textit{Github} (computer code), \textit{Getty Images} (image licensing), \textit{YouTube} (video repositories) or directly through personal websites, the internet has arguably facilitate the most extensive period of information sharing in all of human history.  At the same time, there have also been disputes surrounding the extent to which digitized information could be made available to users. Indeed, controversy has been `part and parcel' of the internet era including major clashes over projects such as \textit{Google Books}\cite{samuelson2009google} and file sharing sites such as \textit{Napster}.\cite{rayburn2001after}  

\subsection{Copyright and A.I. Data Collection}
Although disagreement over proper the scope of copyright is not new, the advent of LLMs has brought with it heightened concerns about the acquisition and use of otherwise copyrighted source materials.\cite{samuelson2023generative}  Virtually all model providers and large scale datasets collected in furtherance of building LLMs have ignored both the website terms of service in the scraping of websites as well as licensing restrictions attached to the respective content collected.\cite{longpre2024large}  As a result, the internet has seen a rise in restrictions upon sharing including a range of efforts to prevent data from being used in the training of A.I. systems.\cite{longpre2024consent}  
Individual creators and content based organizations threatened by A.I. systems that arguably undermine creators future economic prospects have begun to place additional restrictions on their works in an effort to limit reuse, redistribution, and commercial use of their copyrighted materials in the building of A.I systems.\cite{longpre2024consent}

Since the advent of the large-scale public internet, there have been a variety of public and commercial efforts to track its development and growth.\cite{mcmurdo1995internet}\cite{alnoamany2014and}  Much of those efforts centered around ``search'' and helping route individuals to relevant webpages. However, indexing the internet is not the precisely akin to full collection of content.  For example, in the early years of the current millennia, linguists were slow to include large-scale web content given anxiety over copyright in the underlying source materials.\cite{ide2002american}  While some of such materials did eventually make their way into important corpora,\cite{ide2008american} the specter of legal restrictions caused some to limit their use of materials obtained from the internet. 

Other groups, however, were far less motivated by such concerns and began to look at the internet as a premier source of data.  Beyond mere tracking and indexing, internet data has been subjected to various large scale collection efforts including graph data, images, metadata and the underlying text.\cite{buck2014n}\cite{leskovec2016snap}\cite{deng2009imagenet}  \textit{Common Crawl}\cite{smith2013dirt} one of longest standing efforts to collect web-scale data has served as a foundational dataset in many early LLMs. \textit{Common Crawl} and subsequent efforts to build large-scale A.I. training datasets such as the \textit{Colossal Cleaned Common Crawl} (C4)\cite{raffel2020exploring}, \textit{The Pile}\cite{gao2020pile} and \textit{Dolma}\cite{soldaini-etal-2024-dolma} are replete with copyrighted data. 

%Common Crawlhas been cited in thousands of academic articles \footnote{https://github.com/commoncrawl/cc-citations/} 

As noted earlier, the collection and distribution of such materials relies upon ``fair use'' as a justification. ``Fair use'' is fact-specific inquiry meaning that whether a particular use of copyrighted material is considered ``fair use'' depends upon specific details that must be evaluated on a case-by-case basis.  For example, when an academic institution or other non-profit type research organization collects data for research purposes that activity would likely be ``fair use.''  Yet, if that same dataset were deployed for subsequent commercial use by an entity whose direct or indirect aim is to undercut the commercial viability of the original creator (\textit{e.g.} coder, artist, author or musician) that activity might not be characterized as ``fair use.''  Even if ``fair use'' were not deemed to cover a particular usage, it is possible that royalty system including perhaps compulsory licensing might be a vehicle for rewarding creators\footnote{A market based licensing and royalty system including perhaps a compulsory licensing would be more ethical than allowing individuals and organizations to seize the creative works of others without \textit{any} compensation. Such ideas are explored in a recent report released by the U.S. Copyright Office.\cite{Jaffe2025}} while still allowing for innovation in A.I. model building to continue.\footnote{In a letter sent to the White House Office of Science and Technology (OSTP), OpenAI argued that ``[A]pplying the fair use doctrine to AI is not only a matter of American competitiveness — it’s a matter of national security ... If the PRC’s developers have unfettered access to data and American companies are left without fair use access, the race for AI is effectively over.''\cite{OpenAI}  While clarity regarding the legal treatment of this question would be helpful, it is far from clear that the requirement of royalty payments to creators would materially impair the rate of innovation.}

Certain scholars working with datasets such as \textit{Common Crawl}, \textit{C4} and \textit{The Pile} have recognized the looming copyright questions surrounding these efforts.\cite{schafer2016commoncow}\cite{habernal2016c4corpus}  For example, authors of the recently released  \textit{Dolma} dataset stated ``that the legal landscape of A.I. is changing rapidly, especially as it pertains to use of copyrighted materials for training models.''\cite{soldaini-etal-2024-dolma}  However, they still chose to distribute their dataset because the ``sources were publicly available and already being usedin large-scale language model pretraining (both open and closed).'' 

This perspective is emblematic of much of the broader literature on ethics in A.I. where there has been much greater focus on questions of model ‘openness’ and `A.I. alignment' than respect for the scope of the moral and legal rights of creators.  Although issues of transparency and model toxicity are important, they are far from the only consideration worthy of attention.  

Most recently, the \textit{Common Corpus} dataset\cite{arnett2024toxicity}  was released on the \textit{Hugging Face} platform.  The dataset was promising as the authors claimed that the compilation ``contains only data that either is uncopyrighted or permissively licensed.''  Unfortunately, the rhetoric surrounding the dataset does not match its reality.  Although the authors recognize the ethical and potential legal issues associated with scraping data without the consent of the data creator, the authors provide very little description of their copyright audit process.  It turns out that even a cursory inspection of the dataset reveals a significant volume of copyright materials contained therein. 


%As noted earlier, the extant literature has in part begun to acknowledge and confront some of the potential ethical and legal challenges associated with the construction of modern LLMs. Much of the focus, however, has been directed at questions of model ‘openness.’   While ‘openness’ is certainly an attractive property of a development pipeline, it is insufficient without a clear analysis of the ethical and legal requirements.  

%by contrast, does not even pretend to care about the rights of content creators.  Instead, it is includes components such as Books3, Pile-CC, Github and Wikipedia.   FOOTNOTE  [[ The use of Wikipedia (whose pages feature a wide range of licenses selected at the behest of users) is almost certainly problematic at scale.  The Wikimedia foundation lacks the legal authority to change the license terms for  ]]


%As noted earlier, the extant literature has in part begun to acknowledge and confront some of the potential ethical and legal challenges associated with the construction of modern LLMs. Much of the focus, however, has been directed at questions of model ‘openness.’   While ‘openness’ is certainly an attractive property of a development pipeline, it is insufficient without a clear analysis of the ethical and legal requirements.  

%The authors of Common Corpus dataset recognize that it is unethical (and arguably illegal) to scrape (data) without the consent of the data creator.   While the authors claim that the dataset ‘contains only data that is uncopyrighted or permissively licensed,’ the authors provide no description of the process by which such copyright analysis was conducted.  And in fact, even a minimum inspection reveals a significant amount of copyright materials contained with the dataset.   

%The Pile, by contrast, does not even pretend to care about the rights of content creators.  Instead, it is includes components such as Books3, Pile-CC, Github and Wikipedia.   FOOTNOTE  [[ The use of Wikipedia (whose pages feature a wide range of licenses selected at the behest of users) is almost certainly problematic at scale.  The Wikimedia foundation lacks the legal authority to change the license terms for  ]]

   




